
\documentclass[17pt]{beamer}

\usepackage[utf8]{inputenc}
\usepackage[english,russian]{babel}
%\usepackage{tikz}
\hypersetup{colorlinks,linkcolor=,urlcolor=blue}

\usetheme{Warsaw}

\begin{document}

\title{Физиологические факторы риска}
\author{Татьяна Соловьева, Сергей Кирьянов, ...}

\begin{frame}{Физиологические факторы риска: директивы и стандарты}
\begin{center}
    Татьяна Соловьева \\ Сергей Кирьянов \\
    Артем Синкин \\ Александр Сорокин \\
    \vspace{10 mm}
    Tallinn, 2012
\end{center}
\end{frame}

\begin{frame}
\begin{itemize}
\item Директива -- тип законодательного акта ЕС.
\item Она обязывает государство-член в указанный срок принять меры, 
направленные на достижение определенных в ней целей.
\item Директивы дополняются стандартами (EN), содержащими дополнительные детали.
\item Также существуют международные стандарты (ISO).
\end{itemize}
\end{frame}

\begin{frame}
Директивы описывают обязательства работодателя по обеспечению:
\begin{itemize}
\item 89/391/EEC общей безопасности и здоровья работников
\item 89/654/EEC минимальных требований к рабочим местам, в т. ч. свободы передвижения
\item 89/655/EEC + 89/656/EEC 
соответствия работнику оборудования и средств личной защиты 
\end{itemize}
\end{frame}

\begin{frame}
Директивы описывают обязательства работодателя по обеспечению:
\begin{itemize}
\item 90/269/EEC снижения риска травм позвоночника при поднятии тяжестей
\item 90/270/EEC минимальных требований по работе с ЭВМ
\item 93/104/EC организации рабочего времени (монотонность, перерывы, смены, отпуска)
\end{itemize}
\end{frame}

\begin{frame}
Директивы описывают обязательства работодателя по обеспечению:
\begin{itemize}
\item 98/37/EC + 2006/42/EC эргономичности конструкции и безопасности механизмов
\item 2002/44/EC ограничения вибрации
\end{itemize}
\end{frame}

\begin{frame}
\begin{itemize}
\item EN 614: безопасность механизмов; принципы эргономичного дизайна.
\item EN 1005: безопасность механизмов; физическая деятельность человека.
\item prEN 13921: средства личной защиты
\item EN ISO 12100: безопасность механизмов; основные принципы конструирования
\end{itemize}
\end{frame}

\begin{frame}
EN ISO 9241: эргономические требования для офисной работы за монитором
\begin{itemize}
\item EN ISO 9241-4: к клавиатуре
\item EN ISO 9241-5: к расположению оборудования, мебели и рабочей позе
\item EN ISO 9241-9: к устройствам ввода (кроме клавиатуры)
\end{itemize}
\end{frame}

\begin{frame}
\begin{center}
составлено по материалам
\url{http://osha.europa.eu/en/topics/msds/legislation_html} \\
\vspace{10 mm}
Спасибо за внимание.
\end{center}
\end{frame}

\end{document}

%pdflatex -halt-on-error h3.tex